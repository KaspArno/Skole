\documentclass[a4paper,UKenglish]{article}
\usepackage[utf8]{inputenc}
\usepackage{fontenc,url}
\usepackage{babel,textcomp}
\usepackage[round]{natbib}
\usepackage{graphicx}
\usepackage{subcaption}
\graphicspath{ {images/} }
%\urlstyle{sf}

\title{Utvikling av brukergrensesnitt for kartverkets geodatabase}
\author{Kasper Skjegggestad}
\begin{document}
\maketitle
\tableofcontents

\section{Introduction}

%Skal inneholde bakgrunnen for arbeidet, problemets natur og dets signifikans. 
%Svar på hvorfor akkurat dette er interesangt. 
%På en god og klar måte, gjør rede for målene med rapporten
%introduser relevante hypoteser
%Ikke inkluder data eller konklyusjoner fra arbeide som rapporteres

Det er utviklet en geografisk database som representerer tilgjengelighet av kartverket. Måle med databasen er å representere tilgjenelighetsstatus i tettsteder og friluftsområder i Norge. Databasen kan derfor være et nyttig verktøy for arealplanlegging, verdlikehold, statistikk, analyse og som en informasjonskilde for sluttbrukere. All data er lagert i en database, endringer og nyregistreringer blir synkronisert en gang om dagen med en kartlient. Kartklienten viser symboler som representerer status for tilgjenligheten av objektet. I tillegg viser kartklienten alle registrerte målinger og bilder av objektene. For å bruke dataen i databasen, er det nødvendig med kunnskaper om databaser, SQL eller QGIS. Det finnes en WMS kartklient som kan visualisere dataen fra databasen, men det er ikke mulig å kjøre analyse på denne. Det som trengs er et brukergrensesnitt som åpner for at alle kan få tilgang til informasjonen i databasen uten at forhånskunskaper skal være nødvendig.
Hvordan et slikt brukergrensesnitt burde se ut og hvordan det kan utvikles skal bli drøftet her.

\end{document}