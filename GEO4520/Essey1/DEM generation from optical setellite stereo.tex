\documentclass[a4paper,UKenglish]{article}
\usepackage[utf8]{inputenc}
\usepackage{fontenc,url}
\usepackage{babel,textcomp}
\usepackage[round]{natbib}
%\urlstyle{sf}

\title{DEM generation from optical satellite stereo}
\author{Kasper Skjegggestad}
\begin{document}
\maketitle
\tableofcontents

\section{Intro}

Digital elevation models (DEM) are widely used in research by scientists in governmental, university and private organizations. It is particulary usefull for for geosientific applications such as glaciers and rockglaciers, geomorphology and georisk, hydrology, and land cover/ land use \citep{toutin08}. All researches based on DEMs will be derecly dependent on the quality of the DEM.

\section{Main}

Everything we see is in 3D, this is thanks to the fact that we have 2 eyes that catches the lightwawes from the objects in a slightly different angle. Knowing this, we can create 3D images withe the help of 2 cameras, or by taking the picture from two slightly different angels. It is therse priniples that helps us create DEMs with optical satellite stereo. The difference between the two images will help us find the elevation of an object. In principle, we have 2 methodes of achiving stereo photho with satalites. Across-track stereoscopy from two different orbits, and along-track stereoscopy from the same orbit isong one downlooking camera (NADIR) and one backlooking. The across-track stereoscopy will have photos taken og different dates, while the along-track stereoscopy photops are accuired withins secons apart from each other. This makes the images more similar, and therfor more suited for image matching. Image matching is important to find the parallax of objcets in the two different photos. 

\bibliographystyle{apalike}
\bibliography{kilder}

\end{document}
