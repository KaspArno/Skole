\documentclass[a4paper,UKenglish]{article}
\usepackage[utf8]{inputenc}
\usepackage{fontenc,url}
\usepackage{babel,textcomp}
\usepackage[round]{natbib}
%\urlstyle{sf}

\title{DEM generation from optical satellite stereo}
\author{Kasper Skjegggestad}
\begin{document}
\maketitle
\tableofcontents

\section{Intro}

Digital elevation models (DEM), or digital terrain models (DTM) are widely used in research by scientists in governmental, university and private organizations. It is particulary usefull for for geosientific applications such as glaciers and rockglaciers, geomorphology and georisk, hydrology, and land cover/ land use \citep{toutin08}. DTMs can, among other things, be used to find were a landslide might originate, locate precioitaions zones or find the voulm changes if a glacier. All researches based on DTMs will be derecly dependent on the quality of the DTM. The quality of a DTM depends on the method of how it was collected and generated. DTM can be generated with inSAR, LiDAR, and with stereoscopy. The last methot will be disscued here, and are based on \cite{toutin08} and \cite{toutin01}. 

\section{Generating DEM}

Everything we see is in 3D, this is thanks to the fact that we have 2 eyes that catches the lightwawes from the objects in a slightly different angle. Knowing this, we can create 3D images withe the help of 2 cameras, or by taking the picture from two slightly different angels. It is therse priniples that helps us create DEMs with optical satellite stereo. The difference/parallax between the two images will help us find the elevation of an object. In principle, we have 2 methodes of achiving stereo photho with satalites. Across-track stereoscopy from two different orbits, and along-track stereoscopy from the same orbit isong one downlooking camera (NADIR) and one backlooking. The across-track stereoscopy will have photos taken on different dates, while the along-track stereoscopy photops are accuired withins secons apart from each other. This makes the images more similar, and therfor more suited for image matching. Image matching is important to find the parallax of objcets in the two different photos.The process to extrackt ekevation parallax is applied using the image grey-levels, the image features or a hybrid appoach. \cite{toutin01} describes the different processing steps to prouce DEMS from stereo images in broad terms ad follows: 
\begin{enumerate}
\item to acquire teh stereo image data with supplementary information such a ephemeris and attitude data if availeble;
\item to collect Global Controll Points (GCP) to compute or refine teh stereo-model geometry;
\item to extract the elevation parallax;
\item to compute the 3d cartographic co-orinates using 3D stereo-intersection; and
\item to create and post-process the DEM (filtering, 3D editing and moothing).
\end{enumerate}

\section{Acquring data}

ASTER (Advanced Spaceborne Termal Emission and Reflection Radiometer) is a satellitte with both a downward looking sensor, and  backwars looking one. ASTER was sent up to obtain high spatial resolution of the earth. The objectives of the alomg-track stero experiment were: to acuire cloud free stero coverage of 80$\%$ of the Earth' land surface beween 82$^{\circ}$ N and 82$^{\circ}$ S, and to produce, with commercial software. standard prodct DEMs at a rate of one per day \citep{toutin08}. Due to the high temporal resolution provided from ASTER, it's stereophotos are now one of the most used in DEM generation. There are several comersial of the shelf (COTS) softwares for processing stereo ASTER data and for generating DEMS. ASTER produses two types of data, level 1A and level 1B. Level 1A is the preferred data by photogrammetrist, and is the only source for generating the higher-level data products like DEMs and ortho-images. It is anotated with a lot of spacecraft ancillaru information, such as radimometric and geometric coefficients and georeferencing parameters computed and appended. Level 1B are projected to the map using the radiometric calibration and geometric correction coefficeint for resampling.
Global Controll Points (GCP) are collected in order to obtain a cartograpic sandard accuracy. GCPs are collected by finding a point in both image were you know the coorinates and/or elevation. The GCPs should cover the full elevation range of the terraing, and to avoind extrapolation in planimetry, it should be spread at the border of the stereo-pair.

\section{DEM accuracy}

Accuracy refers to the closeness of a measured value to a standard or known value, precission referes to the closness of two or more measurements to each other. When making DEMs, both high accuracu and presision is desired.

\bibliographystyle{apalike}
\bibliography{kilder}

\end{document}
